\documentclass[parskip=half,
 fontsize=12pt, bibtotoc,
 ngerman]
 {article}
%%Präambel
\usepackage[utf8]{inputenc} 
\usepackage[ngerman]{babel}
\usepackage{libertine}
\usepackage[T1]{fontenc}
%\usepackage{geometry}
\usepackage[bottom]{footmisc}
\usepackage{setspace}
\usepackage{anysize}
%Bibliographie
\usepackage{typearea}
\usepackage{biblatex}
\renewcommand{\postnotedelim}{\addcolon\addspace}
%\usepackage[babel,german=guillemets]{csquotes}
\DeclareFieldFormat{postnote}{#1}
\bibliography{abschlussbericht}
\author{Terese Haimberger, Lea Helmers, Jiang Hongliang, Mahmoud Kassem, Daniel Theus, Moritz Walter}
\title{Softwareprojekt: Rekonstruktion metrischer Graphen}
\date{}
\usepackage[left=3cm,right=4cm,top=2cm,bottom=2cm]{geometry}
\setlength{\parindent}{0pt}

\begin{document}
\maketitle
\onehalfspacing
\section{Rekonstruktion metrischer Graphen: der Algorithmus}
Im Rahmen des Softwareprojekts "`Anwendungen effizienter Algorithmen"' haben wir uns damit befasst, einen Algorithmus umzusetsen, der aus einer Punktmenge den zugrundeliegenden Graphen sowie dessen Metrik rekonstruiert. Dadurch soll Struktur in gro{\ss}e Mengen geometrischer Daten gebracht werden, was deren Analyse und Weiterverarbeitung erleichtert. Als Grundlage für unsere Arbeit diente uns das Paper von Aanjaneya et al. \cite{chenEa2012}, welches einen Algorithmus für die Rekonstruktion metrischer Graphen beschreibt und dessen Richtigkeit beweist.\newline
\printbibliography

\end{document}